\defmodule {smultin}

Testing the uniformity and independence of a RNG amounts to testing
that $t$-dimensional vectors $(u_i,\dots,u_{i+t-1})$ of successive
output values of the RNG behave like random points uniformly distributed
over the unit hypercube $[0,1]^t$, for all $t$.
A natural approach for testing this is to generate such vectors and measure
(in some way) the uniformity of their distribution in the unit hypercube.

%%%%%%%%%%%%%
\paragraph*{A class of tests based on the multinomial distribution.} \

One simple way of measuring this uniformity is as follows.\index{multinomial}
For some integer $d$, partition the interval $[0,1)$ into $d$ equal
segments.  This determines a partition of $[0,1)^t$ into $k = d^t$
small hypercubes (or cubic {\em cells\/}) of equal sizes.
Then, generate $n$ random points in the unit hypercube, using $nt$
output values from the generator, and let $X_j$ be the number of
points falling into cell $j$, for $0\le j\le k-1$.
Under $\cH_0$, the vector $(X_0,\dots,X_{k-1})$ has the
{\em multinomial distribution\/} with parameters $(n,1/k,\dots,1/k)$.
The next step is to measure how well the observed vector
$(X_0,\dots,X_{k-1})$ ``agrees'' with this multinomial distribution.
For example, if $n \gg k$, the $X_j$'s should not be too far from
their expected values $E[X_j] = \lambda = n/k$.
The most popular test statistic in this context is Pearson's
chi-square \cite{rKNU98a,sLAW91a,tREA88a}:
\eq
  X^2 = \sum_{j=0}^{k-1} {(X_j - \lambda)^2 \over \lambda}
      = -n + {1\over\lambda} \sum_{j=0}^{k-1} X_j^2,
                                                       \label {eq:X2}
\endeq
Its distribution under $\cH_0$ is approximately
chi-square with $k-1$ degrees of freedom, if $\lambda$ is large enough.
Other test statistics can be used as well, and some of
them turn out to be better than the chi-square for detecting
deficiencies in typical RNGs.

The present module implements several such tests for the
multinomial distribution, and for a variant of it where the points
are formed by overlapping vectors.
These tests are described and studied by
L'Ecuyer, Simard, and Wegenkittl \cite{rLEC02c}.
The test statistic has the general form
\eq
  Y = \sum_{j=0}^{k-1} f_{n,k} (X_j)                   \label{eq:Y}
\endeq
where $f_{n,k}$ is a real-valued function which may depend on $n$ and $k$.
A subclass is the {\em power divergence\/} statistic\index{power divergence}
\eq
  D_\delta = \sum_{j=0}^{k-1} {\frac2{\delta(1+\delta)}} X_j
    \left[\left(\frac{X_j}{\lambda}\right)^\delta -1\right],   \label{eq:Ddelta}
\endeq
studied in \cite{tREA88a},
where $\delta > -1$ is a real-valued parameter and
$\delta=0$ means the limit as $\delta\to 0$.
One has $D_1 = X^2$ as a special case.

Other choices of $f_{n,k}$ are given in Table~\ref{tab:choosef}.
In each case, $Y$ is a measure of clustering: It decreases
when the points are more evenly distributed between the cells.
The loglikelihood\index{loglikelihood} statistic $G^2$ is also a
  special case of $D_\delta$
for $\delta\to 0$ \cite{tREA88a}, and it is related to $H$ via the relation
$H = \log_2(k) - {G^2 / (2n \ln 2)}$.
The statistic $N_b$ counts the number of cells that
contain exactly $b$ points (for $b\ge 0$), $W_b$ is the number of cells that
contain at least $b$ points (for $b\ge 1$),  %%  We need W_1 2 lines below.
and $C$ is the number of collisions\index{collisions}
(i.e., the number of times a point falls in
a cell that already has one or more points in it).
These statistics are related by $N_0 = k - W_1 = k - n + C$,
$W_b = N_b + \cdots + N_n$, and $C = W_2 + \cdots + W_n$.


\begin{table}[htb]
\caption {Some choices of $f_{n,k}$ and the corresponding statistics $Y$.}
\label {tab:choosef}
\smallskip
\centering
\renewcommand {\arraystretch}{1.2}
\begin{tabular}{|rcc|}
\hline
 $Y$        & $f_{n,k}(x)$             &    name  \\
\hline
 $D_\delta$ & $2x[(x/\lambda)^\delta-1]/(\delta(1+\delta)) $
                                       & power divergence \rule{0pt}{16pt}\\
 $X^2$      & $(x-\lambda)^2/\lambda$  & Pearson \\
 $G^2$      & $2x\ln(x/\lambda)$       & loglikelihood \\
 $-H$       & $(x/n)\log_2(x/n)$       & negative entropy \\
 $N_b$      & $I[x = b]$       & number of cells with exactly $b$ points \\
 $W_b$      & $I[x \ge b]$     & number of cells with at least $b$ points \\
 $N_0$      & $I[x = 0]$       & number of empty cells \\
 $C$        & $(x-1)\,I[x>1]$  & number of collisions \\
\hline
\end{tabular}
\end{table}


%%%%%%%%%%%%%
\paragraph*{How to generate the cell numbers.} \

The\index{entropy}\index{Pearson}
 standard way of generating the cell numbers is as described earlier,
by generating one fresh uniform for each coordinate of each point.
That is, $nt$ random numbers $u_0,\dots,u_{nt-1}$ are generated and
the $n$ points are $(u_{ti},\dots,u_{ti+t-1})$, for $i=0,\dots,n-1$.
This is the {\em non-overlapping serial\/} approach used in the
classical {\em serial test\/} \cite{rKNU98a}.
It is the method used by default by the function\index{serial}
{\tt smultin\_Multinomial}.
Here, the points are independent and $(X_0,\dots,X_{k-1})$ has the
{\em multinomial distribution\/} as mentioned earlier.

Another way of producing the cell numbers is as follows:
Generate $nt$ random numbers $u_0,\dots,u_{nt-1}$.
For $i=0,\dots,n-1$, let $v_{ti} = (u_{ti},\dots,u_{ti+t-1})$,
and find which of the $t!$ {\em permutations\/} of $t$ objects would
reorder the coordinates of $v_{ti}$ in increasing order.\index{permutations}
Let $k = t!$ number the $t!$ possible permutations from 0 to $k-1$,
and let $X_j$ be the number of vectors $v_{ti}$ that are reordered
by permutation $j$, for $j = 0,\dots,k-1$.
Here, since the permutations are independent, $(X_0,\dots,X_{k-1})$
has the multinomial distribution with parameters $(n,1/k,\dots,1/k)$.

Yet another method is to examine which coordinate in each $v_{ti}$ has
the {\em largest value\/}, and let $X_j$ be the number of vectors $v_{ti}$
whose largest coordinate is the $j$th.
In this case, $k=t$ and  $(X_0,\dots,X_{k-1})$ is again
multinomially distributed with parameters $(n,1/k,\dots,1/k)$.
\hpierre {This is what is implemented in {\tt smultin\_GenerCellMax},
         with $t$ replaced by $td$. }

To change the method for generating the cells numbers
in {\tt smultin\_Multinomial},\index{cells generation}
it suffices to set the environment variable {\tt smultin\_GenerCell}
to the appropriate function ({\tt smultin\_GenerCellSerial},
{\tt smultin\_GenerCellPermut}, etc.) or to
a user-defined function having the same types of parameters and
that generates cell numbers uniformly and independently.

It is also possible and often advantageous to generate cell numbers
that are {\em dependent}.  This is what happens in the
{\em overlapping serial\/} approach, where only $n$ uniforms
$u_0,\dots,u_{n-1}$ are generated; they are placed in a circle
and each one starts a new vector.  The $n$ points are thus
defined as $v_0 = (u_0,\dots,u_{t-1})$, $v_1=(u_1,\dots,u_t)$, \dots,
 $v_{n-t+1} = (u_{n-t+1},\dots,u_n)$,
 $v_{n-t+2} = (u_{n-t+2},\dots,u_n,u_0)$, \dots,
 $v_{n-1} = (u_{n-1},u_n,u_0,\dots,u_{t-3})$,
 $v_n = (u_n,u_0,\dots,u_{t-2})$.
These points are dependent, because their coordinates overlap,
so $(X_0,\dots,X_{k-1})$ is no longer multinomially distributed in
this case. The function {\tt smultin\_MultinomialOver} uses this approach.

Another set of methods generate the cell numbers from a long string
of ``random'' bits, which are produced $s$ bits at a time by the RNG.
Let $k = 2^L$ be the number of cells, for some integer $L$.
Each cell number is generated by taking $L$ successive bits from
the string, either with or without overlap.
The {\em non-overlapping\/} version requires $nL$ bits
($\lceil nL/s\rceil$ calls to the generator) whereas the
{\em overlapping\/} one requires $n$ bits ($\lceil n/s\rceil$ calls).
The overlapping version operates similarly as the overlapping serial
approach described earlier, except that the uniforms $u_j$ are replaced
by bits.
The functions {\tt smultin\_MultinomialBits} and
{\tt smultin\_MultinomialBitsOver} implement these tests.
% They use {\tt smultin\_GenerCellSerialBits}.
(See also the tests
{\tt sentrop\_EntropyDisc} and {\tt sentrop\_EntropyDiscOver}.)

%%%%%%%%%%%%%%%%%%
\paragraph*{Distribution of $Y$ for the non-overlapping case.} \

We now examine the distribution of the different statistics $Y$ in
Table~\ref{tab:choosef}, under $\cH_0$, assuming that $(X_0,\dots,X_{k-1})$
has the multinomial distribution.
Exact expressions for $E[Y]$ and Var$[Y]$ for this situation are given
in Eqs.\ (2.1) and (2.2) of \cite{rLEC02c}.
In fact, one has $E[Y] = k\mu$ where $\mu = E[f_{n,k}(X_j)]$.
These expressions for the mean and variance are cheap to compute
when $\lambda = n/k$ is small
but become very expensive to compute when $n$ and $\lambda$ are large.
For the case where $\lambda \gg 1$, approximations with $o(1/n)$
error are provided in \cite{tREA88a}, page 65.

The following propositions, taken from \cite{rLEC02c}, provide
approximations for the distribution of $Y$ under various conditions.
Define $\sigma^2_N = \Var[Y]$, $\sigma^2_C = \Var[Y]/(2(k-1))$,
\[
  Y^{(N)} = {Y - k\mu \over \sigma_N},
\]
and
\[
  Y^{(C)} = {Y - k\mu + (k-1)\sigma_C\over\sigma_C}.
\]
Observe that $Y^{(N)}$ has mean 0 and variance 1 (the same as a
standard normal) and that $Y^{(C)}$ has mean $k-1$ and variance $2(k-1)$
(the same as the chi-square random variable with $k-1$ degrees of freedom).
Let $\To$ denote convergence in distribution,
$N(0,1)$ the standard normal distribution, and $\chi^2(k-1)$
the chi-square distribution with $k-1$ degrees of freedom.

\begin {proposition}
\label {prop1}
% \vspace {-8pt}
For $\delta > -1$, under $\cH_0$.
\begin {itemize}
\item[{\rm (i)}]
{\rm [Dense case]} \
If $k$ is fixed and $n\to\infty$, $D_{\delta}^{(C)} \To \chi^2(k-1)$.
%
\item[{\rm (ii)}]
{\rm [Sparse case]} \
If $k\to\infty$, $n\to\infty$, and $n/k \to\lambda_0$ where
$0<\lambda_0<\infty$, then $D_{\delta}^{(N)} \To N(0,1)$.
\end {itemize}
\end {proposition}

The counting variables $N_b$, $W_b$, and $C$ do not obey the
preceding proposition.
For example if $k$ is fixed and $n\to\infty$, eventually $N_b$ becomes
0 and $W_b$ becomes equal to $k$.
If both $k$ and $n$ are large, then each $X_j$ is approximately
Poisson with mean $\lambda$,
so $P[X_j = b] \approx e^{-\lambda} \lambda^b/b!$ for $b\ge 0$.
If $k$ is large and $P[X_j=b]$ is small, $N_b$ is thus approximately
Poisson with mean
\eq
  E[N_b] = {n^b e^{-\lambda} \over k^{b-1} b!}.
                                                         \label{eq:ENb}
\endeq
for $b\ge 0$.  The following proposition is also taken from \cite{rLEC02c}.

\begin {proposition}
\label {prop2}
Under $\cH_0$, suppose $k\to\infty$ and $n\to\infty$,
and let $\lambda_\infty$,
$\gamma_0$, and $\lambda_0$ denote positive constants.
\begin {itemize}
\item[{\rm (i)}]
{\rm [Very sparse case]} \
If $b \ge 2$ and $n^b / (k^{b-1} b!) \to \lambda_\infty$, then
$W_b \To N_b \To$ {\rm Poisson}$(\lambda_\infty)$.
For $b=2$, one also has $C \To N_2$.
%
\item[{\rm (ii)}]
For $b=0$, if $n/k - \ln(k) \to \gamma_0$, then
$N_0 \To$ {\rm Poisson}$(e^{-\gamma_0})$.
%
\item[{\rm (iii)}]
{\rm [Sparse case]} \
If $k\to\infty$ and $n/k \to\lambda_0 > 0$,
for $Y = N_b$, $W_b$, or $C$, one has $Y^{(N)} \To N(0,1)$.
\end {itemize}
\end {proposition}

 The exact distributions of $C$ and $N_0$ under $\cH_0$
 (one has $P(C = c) = P(N_0 = k-n+c)$), for the multinomial setup,
 can be found in Knuth's book
 (see \cite {rKNU98a}, page 71), where an algorithm is also given to
  compute all the non-negligible exact probabilities in time $O(n\log n)$.
  Computing the exact distribution is very slow
  when $n$ is large.
  The probability of having exactly $c$ collisions is given by
  \eq
   P[C = c] = \frac{k (k-1) \ldots (k-n+c+1)}{k^n}
             \left\{{n \atop n-c }\right\}           \eqlabel{fmass-collis}
  \endeq
  where the $\left\{{n \atop k}\right\}$ are the Stirling numbers of
  the second kind \cite{iKNU97a}.
  The expected number of collisions is
  $$
    \mu_c \eqdef E[C]
   = k \left[{\textstyle \frac n k - 1 + \left(1 - \frac1k\right)^n}\right]
   \approx \frac{n^2}{2k}.
  $$

  The current implementation of the test based on $C$ uses the following:
  If $n \le 10^5$, the exact distribution is used,
  else  if $n/k \le 1$, $C$ is approximated by the Poisson
  distribution with mean $\mu_c$,
  else it is approximated by a normal distribution
  with the exact mean and standard deviation.
%  These approximations should give 3 decimal digits of precision for
%  the distribution function, except (possibly) far in the lower tail
%  of the distribution, i.e., when $C\ll \lambda$.
 For two-level tests, the
Poisson approximation for the total number of collisions is used.
\hpierre {In this case, how are the two-level tests performed?}


%%%%%%%%%%%%%%%%%%
\paragraph*{Distribution of $Y$ for the overlapping serial approach.} \

Let $X_{t,j}^\o$ be the number of overlapping vectors $v_i$,
$i=0,\dots,n-1$, falling into cell $j$.
The distributions of the statistics $Y$ are more difficult to analyze
in this case, because of the more complicated dependence relationship
between the $X_{t,j}^\o$.

For $\delta > -1$, let
\eq
 D_{\delta,(t)} = \sum_{j=0}^{k-1} {2\over \delta(1+\delta)} X_{t,j}^\o
          \left[(X_{t,j}^\o/\lambda)^\delta -1\right],  \label{eq:Ddeltat}
\endeq
the power divergence statistic for the $t$-dimensional overlapping
vectors, and define
\eq
  \tilde D_{\delta,(t)} = D_{\delta,(t)}-D_{\delta,(t-1)}.
\endeq
The following is taken from \cite{rLEC02c}
(The case $\delta=1$ was proved long ago by Good \cite{rGOO53a}):

\begin {proposition} \null
{\rm [Dense case]} \ Under $\cH_0$, if $k$ is fixed and $n\to\infty$,
$\tilde D_{\delta,(t)} \To \chi^2(d^t-d^{t-1})$.
\end {proposition}

For the sparse case, where $k,n\to\infty$
and $n/k \to\lambda_0$ where $0<\lambda_0<\infty$,
simulation experiments support the conjecture that
\eq
 \frac{\tilde D_{1,(t)}^2 - (k-k')}{\sqrt{2(k-k')}} \To N(0,1)
\endeq
but this has not been proved.

Marsaglia and Zaman \cite{rMAR93b} speculate that $N_0$ is
approximately normal with mean $k e^{-\lambda}$ and variance
$k e^{-\lambda}(1-3e^{-\lambda})$.
This approximation is reasonably accurate for $2 \le \lambda\le 5$
(roughly), but no longer makes sense when $\lambda$ is too large or
close to zero.
Marsaglia \cite{rMAR85a} calls the test based on $N_0$ with $t=2$ the
{\em overlapping pairs sparse occupancy\/} (OPSO) test
(see {\tt smarsa\_CollisionOver} and {\tt smarsa\_Opso}).

Proposition \ref{prop2} (i) and (ii) probably holds in the overlapping
case as well, but we do not have a formal proof.
Simulation experiments indicate that the Poisson approximation
for $C$ is very accurate for (say) $\lambda < 1/32$, and already
quite good for $\lambda\le 1$, when $n$ is large.

%%%%%%%%%%%%%%%%
\paragraph*{Calling the tests.} \

Four functions are available here to launch a test:
{\tt smultin\_Multinomial} for the case where the cell numbers are
independent of each other (so we have the multinomial distribution)
and {\tt smultin\_MultinomialOver} for the case where the cell numbers
are generated by overlapping $t$-tuples.
In each of these two procedures, the type of approximation that is used
for the distributions of the power divergence statistics is determined
by the boolean parameter {\tt Sparse}: The normal approximation is used
when it is true, and the chi-square approximation otherwise. The
two analog functions for bit tests are {\tt smultin\_MultinomialBits} and
{\tt smultin\_MultinomialBitsOver}.


\iffalse  %%%%%%%%%%%%%%%%%%%%%%%%%%%%%%%%%

Furthermore, we may consider the sparse  case, when the density
 $n/k$ $\le  1$,  or the dense case when the density $n/k > 1$.
 We may also generate the  cells number by different methods.
 If we generate the cells number following the method of the serial test,
 then the {\sl Multinomial} test will be equivalent to
 {\tt sknuth\_Serial}, {\tt sknuth\_SerialSparse}, or {\tt smarsa\_SerialOver}
 (see also {\tt sknuth\_Collision} and  {\tt smarsa\_CollisionOver}).
 If we generate a cell by generating a permutation number, we get the
 test {\tt sknuth\_Permutation} (see also {\tt sknuth\_CollisionPermut}).
 It is possible to use yet other methods to generate the cells number.

The  Cressie-Read statistic \cite{tREA88a} is  defined by
\eq
  2nI^\delta = {2\over \delta(1+\delta)} \sum_{i=1}^k
              X_i \left[\left(X_i \over n p_i\right)^\delta -1\right],
                                                       \eqlabel {powdiv}
\endeq
 where $(X_1,\dots,X_k) \ge 0$ is a vector of observations.

For $p_i = 1/k$ and $\delta = 1$, the exact mean and variance
of ${\cal X}^2 = I^1$ are $\mu = k-1$ and $\sigma^2 = (k-1)(2 - (k-1)/n)$
(e.g., \cite{tHAL37a,tKOE80a}).  %%%  Haldane (1937) and Koehler 1980.
For $\delta \not= 0$, we do not have  simple formulae for
these exact values, but they may be  expressed in terms of expectations
of functions of Poisson random variables \cite{tKOE80a,tREA88a}.
For the case $\delta = 0$, \cite{rLEC96e} give expressions
 computable in $O(n^2)$ for the mean and the variance of
the entropy $\hat H$.

If $k$ is relatively small with respect to $n$,
Read and Cressie \cite[p.67]{tREA88a} suggest using a
version of $2nI^\delta$ corrected for the first two moments,
obtaining thus  a statistic $2nI^\delta_C$ for which the mean
and the variance are $k-1 + O(kn^{-3/2})$ and $2(k-1) + O(k^2 n^{-3/2})$,
respectively, and for which the distribution should be closer to a
 Chi-square law with $k-1$ degrees of freedom.
The  corrected statistic is
$$
 2nI^\delta_C = {2nI^\delta - \mu_C \over \sigma_C},
$$
where
\begin {eqnarray*}
 \mu_C      &=&  (k-1)(1-\sigma) + {\delta-1\over n}\left[
                 {2-3k+t\over 3} + {(\delta-2)(1-2k+t)\over 4}\right], \\
 \sigma_C^2 &=&  1 + {1\over 2n(k-1)}\left[ 2 - 2k - k^2 + t  \right.\\[4pt]
            &&   \hspace{-20pt} + \left. (\delta-1)\left[8 -12k -2k^2 +6t
                 + (\delta-1)(4 -6k -3k^2 + 5t)/3 \\
            &&   \hspace{30pt}  + 2(\delta-2)(1-2k+t)\right]\right],\\
  t         &=&  \sum_{i=1}^k {1\over p_i}.
\end {eqnarray*}


For the case with overlap, let us define this time
   $\nu = (n-t+1)/d^t$ ($\approx$ the density).
   If $n$ (the number of balls) and $d^t$ (the number of cells)
   are very large and of the same order of magnitude,
   the number of collisions is a random variable obeying approximately
   a normal law with mean $\mu \approx d^t (\nu - 1 + e^{-\nu})$
   (this follows from Theorem 2 in \cite{rPER95a}),
   and variance $\sigma^2 \approx d^t e^{-\nu}(1-3e^{-\nu})$,
   as conjectured in \cite{rMAR93a}.
   If $n \ll d^t$, then the number of collisions should be
   approximately Poisson with mean $\mu$, while if
   $\nu$ is somewhat large (e.g., $> 6$), then the
   number of empty cells ($d^t - n + {}$ the number of collisions)
    should be approximately Poisson  with mean $d^t e^{-\nu}$.

\fi  %%%%%%%%%%%%%%%%%%%%%%


%%%%%%%%%%%%%%%%%%%%%%%% \newpage
\bigskip\hrule
\code\hide
/* smultin.h for ANSI C */
#ifndef SMULTIN_H
#define SMULTIN_H
\endhide
#include "gdef.h"
#include "fmass.h"
#include "statcoll.h"
#include "gofw.h"
#include "unif01.h"


#define smultin_MAX_DELTA  8
\endcode
\tab
  Maximal number of distinct values of $\delta$ for which the
  statistic $D_\delta$ can be computed in a single test
  (with $\delta=-1$ representing the family of collision-type tests).
\endtab
\code


#define smultin_MAXB  10
\endcode
\tab
  Maximal value of $b$ for which the statistic $W_b$
  can be computed (see Table \ref{tab:choosef}).
%, together with their $p$-values.
% for which we may calculate the
%  statistics {\tt smultin\_WbUrn} and their $p$-values.
\endtab


% \newpage
%%%%%%%%%%%%%%%%%%%%%%%%%%%%%%%%%%%%%%%%%%
\guisec{Environment variables}

The parameters in {\tt smultin\_Envir} are environment variables that
should be fixed once for all and will not normally be changed.

\code

typedef struct {

   double Maxk;
\endcode
\tabb
   Maximal value of $k = d^t$, the number of cells. The default value is
   either $2^{63}$ if 64-bit integers are available, or else $2^{53}$,
   the largest number of cells that can be enumerated by {\tt double's}
   without loss of precision.
\endtabb
\ifdetailed
\code

   double SeuilHash;
\endcode
\tabb
  When the number of cells $k$ is larger than this threshold and
  the average number of balls per cell is less than 1,
  the variable {\tt Hashing} is set to {\tt true} and
   hashing is used to compute the numbers $X_j$ and the power divergence
  statistics.
  In this case, {\tt Count[i]} counts how many $X_j$ are
  equal to $i$, for $i \ge 0$.
  The  default value is {\tt SeuilHash} $= 2^{20}$.
\endtabb
\code

   double HashLoad;
\endcode
\tabb
  Maximal load factor when  hashing is used. The number
  of cells times {\tt HashLoad} is always larger than the number
  of balls. The default value is 0.75. The actual load factor will
  most often be smaller than this value.
\endtabb
\code

   double SeuilEColl;
\endcode
\tabb
  For the two-level collision  tests, if the expected number of
  collisions for a replication is larger than this value, the statistic
  {\tt gofw\_Mean} will
  correspond to a test based on the average number of collisions
   using the  normal  approximation.  Otherwise, it will
  correspond to a  test  based on the total number of collisions
   using the  Poisson approximation.
  In both cases, {\tt gofw\_KS} and {\tt gofw\_AD} will
  correspond to  two-level tests as usual, assuming that the number
  of collisions
  $C$ is a random variable approximatively continuous, which is reasonnable
  when the normal approximation is valid.
\endtabb
\code

   double SeuilCOverDense;
   double SeuilCOverNorSup;
   double SeuilCOverNorInf;
   double SeuilCOverSparse;
\endcode
\tabb
  Thresholds determining the different approximations used for
  the distribution function in the {\tt CollisionOver} test.
  Let the density $\nu = n/k$, where $n$ is
  the number of balls thrown into $k$ cells.
  When {\tt SeuilCOverNorSup} $< \nu < $
  {\tt SeuilCOverDense},
  we use the Poisson approximation for the number of empty cells.
  When {\tt SeuilCOverNorInf} $< \nu \le $
  {\tt SeuilCOverNorSup}, we use  the
   normal approximation for the number of collisions.
   For $ \nu < $ {\tt SeuilCOverSparse},  we use  the
   Poisson  approximation  for the number of collisions.
   The  default values are {\tt SeuilCOverDense} = 12,
  {\tt SeuilCOverNorSup} = 5, {\tt SeuilCOverNorInf} = 2,
  {\tt SeuilCOverSparse} = 1.
\endtabb
\fi
\code

} smultin_Envir;


extern smultin_Envir smultin_env;
\endcode
\tab
   This is the environment variable used to keep the values of the fields
   described above in {\tt smultin\_Envir}.
\endtab



%%%%%%%%%%%%%%%%%%%%%%%%%%%%%%%%%%%%%%%%%%
\guisec{Functions to generate the cell numbers}

\code

typedef uint64_t smultin_CellType;
\endcode
  \tab Type used to enumerate cell numbers, a 64-bit integer.
  \endtab
\code


typedef smultin_CellType (*smultin_GenerCellType) (unif01_Gen *, int, int,
                                                   long);
\endcode
  \tab
   Type of function used to generate a cell number. It returns the cell
   number.\index{cells generation}
  \endtab
\code


smultin_CellType smultin_GenerCellSerial (unif01_Gen *gen, int r, int t,
                                          long d);
\endcode
 \tab  Generates and returns a cell number in $\{0,\dots,k-1\}$ for the
  multinomial test, with $k = d^t$. The function generates $t$ integers
  $y_0,\dots,y_{t-1}$ in $\{0,\dots,d-1\}$, using the most significant
  bits of $t$ successive uniforms
  (after throwing away their leading $r$ bits).
  The cell number $c = y_0 d^{t-1} + \cdots + y_{t-2} d + y_{t-1}$
  is returned.
 \endtab
\code


smultin_CellType smultin_GenerCellSerial2 (unif01_Gen *gen, int r, int t,
                                           long d);
\endcode
 \tab  Similar to {\tt smultin\_GenerCellSerial}, except that the
   cell number is   $c = y_{t-1} d^{t-1} + \cdots + y_{1} d + y_0$.
   This is equivalent to using {\tt smultin\_GenerCellSerial},
   because it only changes the numbering of the cells and the power
   divergence test statistics do not depend on the numbering.
 \endtab
\code


smultin_CellType smultin_GenerCellPermut (unif01_Gen *gen, int r, int t,
                                          long junk);
\endcode
 \tab Similar to {\tt smultin\_GenerCellSerial}, except that there are
  $t!$ cells, corresponding to the $t!$ permutations of $t$ real numbers,
  The function generates $t$ uniforms and returns the number of the
  permutation that corresponds to their ordering.
 \endtab
\code


smultin_CellType smultin_GenerCellMax (unif01_Gen *gen, int r, int t,
                                       long junk);
\endcode
 \tab  Similar to {\tt smultin\_GenerCellSerial}, except that $k=t$
  and the cell number is the number of the largest coordinate in $v_{ti}$.
 \endtab
\hide
\code


smultin_CellType smultin_GenerCellSerialBits (unif01_Gen *gen, int r, int s,
                                              long L);
\endcode
 \tab  Generates a cell number in $\{0,\dots,k-1\}$ for the
  multinomial test, from a bit string, where $k = 2^L$.
  The function extracts $s$ bits from each call to the RNG
  (after throwing away their leading $r$ bits)
  and concatenates these bits into a long string.
  Each cell number is generated by taking $L$ successive bits from
  the string, say $b_0,\dots,b_{L-1}$.
% , either with or without overlap.
  The cell number $c = b_0 2^{L-1} + \cdots + b_{L-2} 2 + b_{L-1}$
  is returned.
 \endtab
\endhide

%%%%%%%%%%%%%%%%%%%%%%%%%%%%%%%%%%%%%%%%%%
\guisec{Test parameters having default values}
\nopagebreak

The parameters in the following global structure {\tt smultin\_Param}
often remain the same in a given experiment.
They can practically be viewed as environment variables.
To change their default values, one should create a {\tt smultin\_Param}
structure and pass it as a pointer to the tests.
When a {\tt NULL} pointer is passed as argument to a test, the
default values are used instead.

\code


typedef struct {

   int NbDelta;
   double ValDelta [smultin_MAX_DELTA];
\endcode
 \tabb
  The number of values of $\delta$, and the list of values of $\delta$,
  for which the multinomial tests are applied (simultaneously)
  when calling a test.  If {\tt NbDelta} = $m$, then the
  tests are applied for the $m$ values $\delta_0,\dots,\delta_{m-1}$
  which must be given in array {\tt ValDelta[0..m-1]}.
  The value $\delta = -1$ corresponds to the collision test
  and the other values of $\delta$ correspond to power divergence tests.
  The default values are $m=2$, $\delta_0 = -1$, $\delta_1 = 1$.
 \endtabb
\code

   smultin_GenerCellType GenerCell;
\endcode
\tabb
  This function is used to generate the cell numbers in the tests.
  It is one of the function {\tt smultin\_GenerCell} described above.
  The default value is {\tt smultin\_GenerCellSerial}.
\endtabb
\code

   int bmax;
\endcode
 \tabb  In the non-overlapping case for $\delta = -1$, in addition to
   applying the collision test, the statistic $W_b$ and its $p$-value
   are computed for $b=0,\dots,${\tt bmax}.
   For $b=0$, the test is actually based on $N_0$, the number of
   empty cells.
   If the value is negative, none of these statistics is computed.
   The default value is {\tt bmax = -1}.
%  Setting {\tt bmax} to a larger value permits one to see how many cells
%  are empty, how many contain exactly one point, how many contain
%  two points, etc.
 \endtabb
\code

} smultin_Param;


smultin_Param * smultin_CreateParam (int NbDelta, double ValDelta[],
                                     smultin_GenerCellType GenerCell,
                                     int bmax);
\endcode
 \tab
  Function creating and returning a structure that will hold the
  parameters described above.  The parameters have
  the same meaning as in the structure {\tt smultin\_Param}.
 \endtab
\code


void smultin_DeleteParam (smultin_Param *par);
\endcode
 \tab
  Procedure freeing the memory allocated by {\tt smultin\_CreateParam}.
 \endtab

\hide
\code

extern smultin_Param smultin_ParamDefault;
\endcode
\endhide



\ifdetailed  %%%%%%%
%%%%%%%%%%%%%%%%%%%%%%%%%%%%%%%%%%%%%%%%%%
\guisec{Test results}

 The following large structure type {\tt smultin\_Res} is used to
 keep the intermediate and final results of a test.
 It contains the counters, the test statistics, their $p$-values,
 and some quantities related to the theoretical distributions.
 If one needs these results for post-processing or for any
 other reason after a test has ended, it suffices to create  a
 {\tt smultin\_Res} structure by calling the function
 {\tt smultin\_CreateRes}  and pass it as a pointer to the tests.
 Otherwise, a {\tt NULL} pointer can be passed in place of the {\tt res}
 argument to the tests.

\code

typedef enum {
   smultin_CollNotInit,           /* Not initialized */
   smultin_CollExact,             /* Exact distribution */
   smultin_CollNormal,            /* Normal approximation */
   smultin_CollPoissonSparse,     /* Poisson approximation: sparse case */
   smultin_CollPoissonDense       /* Poisson approximation: dense case */
} smultin_CollApproxType;
\endcode
  \tab
   Approximation used for the distribution function in the
       {\tt Collision} and  {\tt CollisionOver} tests.
  \endtab
\code


typedef struct {

   bool Hashing;
\endcode
\tabb
  {\tt true} if hashing is used.
  The value of this variable is chosen for each  test depending on
   {\tt smultin\_env.SeuilHash}.
\endtabb
\code

   smultin_CellType NbCellsTotal;
\endcode
\tabb
   Total number of cells for a test (value of $k$).
 \endtabb
\code

   bool Over;
\endcode
\tabb {\tt true} if we are doing the overlapping version of the tests,
      {\tt false} otherwise.
\endtabb
\code

   smultin_CollApproxType CollApprox;
\endcode
\tabb Approximation used for the  distribution function in the
    {\tt Collision} or {\tt CollisionOver} tests.
 \endtabb
\code

   double Mu [smultin_MAX_DELTA];
   double Sigma [smultin_MAX_DELTA];
\endcode
\tabb
   Theoretical mean and standard deviation (\ref{eq:smultin-mu}) and
   (\ref{eq:smultin-sigma2}), for each value of {\tt ValDelta[s]}.
\endtabb
\code

   double EsEmpty;
\endcode
\tabb
  The expected number of empty cells.
\endtabb
\code

   long CountSize;
   long Count1Size;
   long *Count;
   long *Count1;
   smultin_CellType *Cell;
   smultin_CellType *Cell1;
\endcode
\tabb
  {\tt CountSize} is the size of array {\tt Count} and
  {\tt Count1Size} is the size of array {\tt Count1}.  These arrays contain
  counters keeping track of the number of balls in a cell, numbered
  by the array index in the case where {\tt Hashing = false}.
  If {\tt Hashing = true}, the counter with index $i$
  indicates the number of balls in cell numbered by  {\tt Cell[i]}.
  {\tt Count1} and  {\tt Cell1} are used in
  the {\tt Over} version of the  tests in dimension $t-1$.
\endtabb
\code

   long NbSize;
   long Nb1Size;
   smultin_CellType *Nb;
   smultin_CellType *Nb1;
\endcode
\tabb
  {\tt NbSize} is the size of array {\tt Nb} and
  {\tt Nb1Size} is the size of array {\tt Nb1}.
  {\tt Nb[i]} is the observed number of cells containing $i$ points.
  {\tt Nb1[i]} is the observed number of cells  containing $i$ points
  in dimension $t-1$ (for the {\tt Over} versions of the tests).
\endtabb
\code

   smultin_CellType NbCells [smultin_MAXB + 1];
   double EsCells [smultin_MAXB + 1];
   smultin_CellType WbCells [smultin_MAXB + 1];
\endcode
\tabb
  The array element {\tt NbCells[i]} is the observed  numbers of
  cells containing exactly $i$ points, where the counters are added over
  {\em all $N$ replications\/} (in contrast with {\tt Nb} and {\tt Nb1},
  which are reinitialized to zero after each replication).
  The test statistics are the sums of the $N$  Poisson variables
  over the $N$ replications.
  {\tt EsCells[i]} and {\tt WbCells[i]} contain the expected number
  (asymptotically when the density $n/k \rightarrow 0$) and the
  observed number of cells containing at least $i$ points, for $i\ge 0$,
  respectively.
\endtabb
\code

   double NbCollisions;
\endcode
\tabb
  Total observed number of collisions, over all $N$ replications.
 \endtabb
\code

   statcoll_Collector *Collector [smultin_MAX_DELTA];
\endcode
\tabb
  {\tt Collector[s]} is used to keep the $N$ observations of $Y$
  for the test based on {\tt ValDelta[s]}.
\endtabb
\code

   gofw_TestArray sVal2 [smultin_MAX_DELTA];
   gofw_TestArray pVal2 [smultin_MAX_DELTA];
\endcode
 \tabb
  After a call to a test ({\tt smultin\_Multinomial},
  {\tt smultin\_MultinomialOver}, etc.) with $N > 1$,
  the values and the $p$-values of the different statistics (but not
  {\em Collisions\/}) for {\tt ValDelta [s]} will be found in arrays
  {\tt sVal2 [s, $\cdot$]} and {\tt pVal2 [s, $\cdot$]}.
  For $N > 1$, the indices {\tt [s, gofw\_Mean]} and {\tt [s, gofw\_Cor]}
  stand for tests based on the empirical mean and correlation
  of the $N$ observations and their $p$-values are computed by assuming
  that they are normally distributed.
  In the cases where $N=1$, {\tt sVal2 [s, gofw\_KSP]} and
  {\tt pVal2 [s, gofw\_KSP]} will contain
  the $p$-value of the test at the first level  and the
  other array elements are unused.
  For the collision tests using the Poisson approximation,
  or using the exact law with $N = 1$,
  {\tt gofw\_Mean} corresponds to a  test based on the total number of
  collisions (or empty cells, in the dense case)
  for the $N$ replications.
 \endtabb
\code

   double pCollLeft, pCollRight;
\endcode
 \tabb The left and the right $p$-values for the collision test.
 \endtabb
\code

   double pColl;
\endcode
 \tabb The $p$-value for the collision test.
 \endtabb
\code

   double pEmpty;
\endcode
\tabb
  The $p$-value for the number of empty cells.
\endtabb
\code

   double pWb [smultin_MAXB + 1];
\endcode
 \tabb
 The  $p$-values for $W_b$, the number of cells containing at
  least $b$ points (see the description of {\tt EsCells} and
  {\tt  WbCells}).
 \endtabb
\fi  %%%%%%%%%%%%%%%%%
\hide  %%%%%%%%%%
\code

   int NbDeltaOld;
\endcode
\tabb
   The number of values of $\delta$ of the previous test (see the
   documentation in {\tt smultin\_Param}). This is necessary to prevent
   memory leaks and bugs, in case {\tt NbDelta} changes from the previous
   test with the same variable {\tt res}.
\endtabb
\code

   double *TabFj[smultin_MAX_DELTA];
\endcode
\tabb If we replicate a test many times, we may keep some complicated
  terms in these tables instead of recomputing them every time
  (depending on the value of {\tt flagTab}).
\endtabb
\code

   int nLimit;
\endcode
\tabb  The tables {\tt TabFj} are precomputed for elements {\tt
  [0..nLimit]}.
\endtabb
\code

   bool flagTab;
\endcode
\tabb This flag is set to true if we precompute tables of values and keep
   them in arrays {\tt TabFj}.
\endtabb
\endhide  %%%%%%%%%%%%
\ifdetailed  %%%%%%%%%%%%%%
\code

} smultin_Res;


smultin_Res * smultin_CreateRes (smultin_Param *par);
\endcode
 \tab
  Creates and returns a structure to hold the results of a test,
  based on the parameter values given in the structure {\tt par}.
  If {\tt par = NULL}, the default parameter values are used
  (see the description of {\tt smultin\_Param}). The user may change
  the parameter values before a test or between tests using the same
  {\tt smultin\_Res} structure.
 \endtab
\code


void smultin_DeleteRes (smultin_Res *res);
\endcode
 \tab
  Recovers the memory allocated to {\tt res} by {\tt smultin\_CreateRes}.
 \endtab



%%%%%%%%%%%%%%%%%%%%%%%%%%%%%%%%%%%%%%%%%%
\guisec{Specialized functions used inside the tests}

These functions are used to compute the theoretical mean, variance,
and correction factors used in some tests, and to compute individual terms
of the test statistics.

\code

typedef double (*smultin_MNTermeType) (double, double, long);
\endcode
  \tab Represents one term for the different
   statistics computed by {\tt smultin\_MultinomMuSigma}.
   The first two arguments are parameters;
   the third is  the variable of the function, equal for example to the
   number of points in a cell.  See {\tt smultin\_MNTermeColl,
   smultin\_MNTermeKhi2, smultin\_MNTermeLogLikhood}, and
   {\tt smultin\_MNTermePowDiv} below.
  \endtab
\code


double smultin_MNTermeColl (double, double, long j);
\endcode
 \tab Returns $\max(0, j-1)$, the number of collisions in a cell
  containing $j$ points.
 \endtab
\code


double smultin_MNTermePowDiv (double Delta, double E, long j);
\endcode
 \tab Returns one term of the {\sl power divergence} statistic:
%  (with $\delta =$ {\tt Delta}):
  $${2j\over \delta(1+\delta)}
      \left[\left(j \over E\right)^\delta -1\right].$$
 \endtab
\code

double smultin_MNTermeKhi2 (double, double E, long j);
\endcode
 \tab Returns one term of the {\em Pearson's chi-square\/} statistic:
%  (with $E =$ {\tt NbEsp})
   $ (j - E)^2 / E$.
 \endtab
\code


double smultin_MNTermeLogLikhood (double, double E, long j);
\endcode
 \tab Returns one term of the {\em loglikelihood ratio\/} statistic:
%  (with $E =$ {\tt NbEsp})
    $ 2 j\; \ln \left(j / E\right)$.
 \endtab
\code


void smultin_MultinomMuSigma (long n, double k, double theta1,
                              double theta2, smultin_MNTermeType F,
                              double *Mu, double *Sigma);
\endcode
 \tab
  Computes the exact values of the mean $\mu$ and the standard deviation
  $\sigma$ of a statistic of the form
%  (with  $T1 = \theta_1$ and $T2 =\theta_2$)
  $$
    Y = \sum_{j=0}^{k-1} F(\theta_1, \theta_2, X_j),
  $$
  assuming that $(X_0,\ldots, X_{k-1})$
  obeys a multinomial law with parameters $(n, 1/k, \ldots,$ $1/k)$.

  If we write $F(X)$ for $F(\theta_1,\theta_2, X)$ (to simplify the notation),
  the values of $\mu$ and $\sigma$ are given by:
  \begin{eqnarray}
    k\mu_0 & \eqdef & \mu\ = \ \sum_{x=0}^n {n \choose x}
      \frac{(k-1)^{n-x}}{k^{n-1}} F(x) \label{eq:smultin-mu}  \\
   \sigma^2 &=& E\left[\left(\sum_{j=0}^{k-1}\left(F(X_j) - \mu_0 \right)
     \right)^2 \right]\\[6pt]
     &=& k E\left[\Big(F(X_0) - \mu_0 \Big)^2 \right] + k(k-1)
    E\left[\left(F(X_0) - \mu_0 \right) \left(F(X_1) - \mu_0 \right) \right].
                                              \label{eq:smultin-sigma2}
  \end{eqnarray}
\iffalse %%%%
  We meet such multinomial  vectors in the following context:
  We throw $n$ balls into $k$ cells, independently and with a
  uniform probability $1/k$ for each cell, and we define
  $X_j$ as the number of balls falling into  cell $j$.
\fi  %%%%
 \endtab
\code

void smultin_PowDivMomCorChi (double Delta, long n, double k,
                              double *MuC, double *SigmaC);
\endcode
 \tab
  Computes and returns the values of $\mu_C = $ {\tt MuC} and
  $\sigma_C = $ {\tt SigmaC} for the two-moment corrected statistic
  $D_\delta^{(C)} = (D_\delta - \mu_C) / \sigma_C$,
  in the cases where we use the approximation by the {\em chi-square\/}
  distribution.
  If $\delta \le -1$ (the collision test), this function computes nothing.
  Here, $n$ is the number of points and $k$ is the number of cells.
 \endtab
\code


void smultin_PowDivMom (double Delta, long n, double k,
                        double NbExp, double *Mu, double *Sigma);
\endcode
 \tab
  Compute the mean $\mu = $ {\tt Mu} and the standard deviation $\sigma = $
  {\tt Sigma} for the  theoretical distribution of the test statistic  for
  this  $\delta = $ {\tt Delta}.  This  function is called when we use the
  normal approximation and for the collision test with the normal
  approximation.
  The returned  values are close to the exact values in most cases.
 \endtab

\fi  %%%%%%%%%%%%%%%%

\ifdetailed  %%%

% \newpage
%%%%%%%%%%%%%%%%%%%%%%%%%%%%%%%%%%%%%%%%%%
\guisec{Distribution for collision test}

This section provides tools to compute the theoretical distribution
of the number of collisions for the collisions test (the non-overlapping
case only).
To use these tools, one must first call {\tt smultin\_CreateCollisions}
to create the appropriate structure.  When it is no longer needed,
this structure can be deleted by
calling {\tt smultin\_DeleteCollisions}.

\code


fmass_INFO smultin_CreateCollisions (long n, smultin_CellType k);
\endcode
 \tab Creates and returns a structure that contains the mass and
   distribution functions of the random variable $C$,
   the number of collisions when $n$ balls (or points)
   are thrown at random and independently into $k$ boxes (or cells), each
   ball having probability $1/k$ of falling into any given box.
   A collision occurs when a ball falls into a box already occupied
   with one or more balls.
   Such a structure is needed for calling
   {\tt smultin\_FDistCollisions}, {\tt smultin\_FBarCollisions} or
   {\tt smultin\_CollisionsTerm}. It can be deleted by calling
   {\tt smultin\_DeleteCollisions}.
  (The module {\tt fmass} is in library ProbDist.)

 \endtab
\code


void smultin_DeleteCollisions (fmass_INFO W);
\endcode
  \tab Deletes the structure {\tt W} created previously
   by {\tt smultin\_CreateCollisions}.
 \endtab
\code


double smultin_FDistCollisions (fmass_INFO W, long c);
\endcode
 \tab Returns $P[C\le c]$ for the random variable $C$
  described in {\tt smultin\_CreateCollisions}, from the structure
  {\tt W}, which must have been created previously by calling
  {\tt smultin\_CreateCollisions} with the desired $n$ and $k$.
 \endtab
\code


double smultin_FBarCollisions (fmass_INFO W, long c);
\endcode
 \tab  Similar to {\tt smultin\_FDistCollisions}, but returns $P[C\ge c]$.
\endtab
\code


double smultin_CollisionsTerm (fmass_INFO W, long c);
\endcode
 \tab  Similar to {\tt smultin\_FDistCollisions}. Returns an
   approximation of $P[C = c]$ given in (\ref{fmass-collis}).
 \endtab

\fi



% \newpage
%%%%%%%%%%%%%%%%%%%%%%%%%%%%%%%%%%%%%%%%%%
\guisec{The tests}

% \resdef

\code

void smultin_Multinomial (unif01_Gen *gen, smultin_Param *par,
   smultin_Res *res, long N, long n, int r, long d, int t, bool Sparse);
\endcode
 \tab
  This function applies the power divergence
  test,\index{Test!Multinomial} based on
%  (corrected for the first two moments)
  statistic $D_\delta$, for each value of $\delta$ specified in the array
  {\tt par->ValDelta[0..NbDelta-1]} (where $\delta = -1$
  corresponds to the collision test, based on $C$).
  The theoretical distribution of $D_\delta$ (with a two-moment correction)
  is approximated by a normal if {\tt Sparse = true}, and by a chi-square
  with $k-1$ degrees of freedom if {\tt Sparse = false}.
  This function also applies tests based on $W_b$ for
  $b=0, 2,\dots, ${\tt par->bmax}.
  All these tests are applied simultaneously, using the same cell
  countings. The cell numbers are generated by the function in variable
  {\tt par->GenerCell}.
  By default, it is {\tt smultin\_GenerCellSerial}.
  Both {\tt par} or {\tt res} can be set to the {\tt NULL} pointer, in
  which case these structures are created and deleted internally.
  It is recommended to have $n/k > 8$ if {\tt Sparse = false},
  except for $\delta=1$.
  If {\tt Sparse = true}, $n$ and $k$ should both  be very large.
  Restriction: $k = d^t \le ${ \tt smultin\_Maxk}.
 \endtab
\code


void smultin_MultinomialOver (unif01_Gen *gen, smultin_Param *par,
   smultin_Res *res, long N, long n, int r, long d, int t, bool Sparse);
\endcode
 \tab
  Similar to {\tt smultin\_Multinomial}, but where the $n$ cell
  numbers are generated using the {\em overlapping serial\/} approach,
  as described earlier in the paragraph ``How to generate the cell numbers''
  and used in  {\tt smarsa\_SerialOver}.\index{Test!MultinomialOver}
%  The $n$ generated values are placed in a circle, and the $n$ $t$-tuples
%  of successive values are the points.
 \endtab
\code


void smultin_MultinomialBits (unif01_Gen *gen, smultin_Param *par,
   smultin_Res *res, long N, long n, int r, int s, int L, bool Sparse);
\endcode
 \tab
  Similar to {\tt smultin\_Multinomial}, except that the cells are generated
  from a string of bits obtained by taking {\tt s} bits from each
  output value.\index{Test!MultinomialBits}  There are $k = 2^L$ cells
  and each cell number is determined by taking $L$ successive bits from
  the string.
%  The cell numbers are generated via {\tt smultin\_GenerCellSerialBits}.
% \richard {Non. Quand $L\ge s$, le test appelle {\tt smultin\_Multinomial},
% et quand $L < s$, on ne peut pas l'utiliser, car on g\'en\`ere plus d'un
% point avec chaque U01. Cette fonction n'est pas utilis\'ee.}
  In the case where $L = ts$ for some integer $t$, this test is
  equivalent to {\tt smultin\_Multinomial} with $d = 2^s$ and $t = L/s$.
  The present function is to cover the other cases (e.g., if $L < s$).
  Restrictions:  $L \bmod s = 0$ when $L > s$, and
   $s \bmod L = 0$ when $s > L$.
 \endtab
\code


void smultin_MultinomialBitsOver (unif01_Gen *gen, smultin_Param *par,
   smultin_Res *res, long N, long n, int r, int s, int L, bool Sparse);
\endcode
 \tab
  Similar to {\tt smultin\_MultinomialBits}, except that the $n$ cell
  numbers are generated using the {\em overlapping\/} approach at the
  bit level.  The $n$ bits are placed in a circle and each block of $L$
  successive bits determines a cell number.\index{Test!MultinomialBitsOver}
  $L$ and $s$ do not have to divide each other.
 \endtab
\code
\hide
#endif
\endhide
\endcode
