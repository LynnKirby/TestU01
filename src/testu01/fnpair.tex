\defmodule {fnpair}

This module applies close-pairs tests from the module {\tt snpair}
to a family of generators of different sizes.

\bigskip
\hrule
\code\hide
/* fnpair.h  for ANSI C */
#ifndef FNPAIR_H
#define FNPAIR_H
\endhide
#include "TestU01/gdef.h"
#include "TestU01/ffam.h"
#include "TestU01/fres.h"
#include "TestU01/fcho.h"
#include "TestU01/snpair.h"


extern long fnpair_Maxn;
\endcode
\tab
  Upper bound on $n$.
  When $n$ exceeds its limit value, the test is not done.
  Default value: $n = 2^{22}$.
\endtab
\ifdetailed  %%%%%%%%%%


%%%%%%%%%%%%%%%%%%%%%%%%%%%%%%%%%%%%%%%%%%
\guisec{Structure for test results}

The test results for the tests in this module can be recovered
in the following structure.

\code

typedef struct {
   ftab_Table *PVal[snpair_StatType_N];
} fnpair_Res1;
\endcode
 \tab
  Structure for keeping the tables of results of the tests in
  this module. Depending on the test, not all elements of the array
  will be meaningful; those elements that are not will be set to
  the NULL pointer.
 \endtab
\code


fnpair_Res1 * fnpair_CreateRes1 (void);
\endcode
 \tab
  Creates and returns a structure that will hold the results
  of the tests in this module.
 \endtab
\code


void fnpair_DeleteRes1 (fnpair_Res1 *res);
\endcode
 \tab
  Frees the memory allocated by {\tt fnpair\_CreateRes1}.
 \endtab

\fi    %%%%%%%%%%%%%%%


%%%%%%%%%%%%%%%%%%%%%%%%%%%%%%%%%%%%%%%%%%
\guisec{Choosing the parameter $\mathbf{m}$}

 One may choose to vary the parameter $m$ as a function of the sample size
in the test {\tt fnpair\_ClosePairs1}. In that case, one must create
 an appropriate Choose function, pass it as a member of the argument {\tt
cho} to the test,  and call the test with a negative $m$ to signal that
this is the case. If one wants to
 do the test with a fixed $m$, one has but to give the given $m> 0$ as a
parameter to the test.

\code


fcho_Cho *fnpair_CreateM1 (int maxm);
\endcode
 \tab
  Given the number of replications $N$  and the sample
  size $n$, the parameter $m$ in {\tt snpair\_Close\-Pairs} will be chosen
  by the relation
   $m = \min\left\{{\tt maxm},\sqrt{{n}/\left(4\sqrt{N}\right)}\right\}$.
  If this method
  of choosing $m$ is used, then  the returned structure must be passed as
  the second pointer in the argument {\tt cho} of the test and the
  argument $m$ of the test  {\tt fnpair\_ClosePairs1} must be negative.
 \endtab
\code


void fnpair_DeleteM1 (fcho_Cho * cho);
\endcode
 \tab
  Frees the memory allocated by {\tt fnpair\_CreateM1}.
 \endtab



%%%%%%%%%%%%%%%%%%%%%%%%%%%%%%%%%%%%%%%%%%
\guisec{Applying the tests}

\code


void fnpair_ClosePairs1 (ffam_Fam *fam, fnpair_Res1 *res, fcho_Cho2 *cho,
                         long N, int r, int t, int p, int m,
                         int Nr, int j1, int j2, int jstep);
\endcode
\tab
 This function calls the test {\tt snpair\_ClosePairs}
 with parameters $N$ and $r$, in dimension $t$, with $L_p$ norm,
 sample size $n =$ {\tt cho->Chon->Choose(param, $i$, $j$)},
 for the first {\tt Nr} generators of family {\tt fam}, for $j$ going from
 {\tt j1} to {\tt j2} by steps of {\tt jstep}. The parameters in {\tt param}
 were set at the creation of {\tt cho} and $i$ is the lsize of the
 generator being tested. If $m > 0$, it will be used as is in the test.
 If  $m < 0$, it will be chosen by $m =$
  {\tt cho->Chop2->Choose(param, $N$, $n$)}.
 When $n$ exceeds {\tt fnpair\_Maxn} or $n < 4m^2 N^{1/2}$, the test
 is not done.
\endtab
\code


void fnpair_Bickel1 (ffam_Fam *fam, fnpair_Res1 *res, fcho_Cho *cho,
                     long N, int r, int t, int p, bool Torus,
                     int Nr, int j1, int j2, int jstep);
\endcode
\tab
 Similar to {\tt fnpair\_ClosePairs1} but with {\tt snpair\_BickelBreiman}.
 There is no parameter $m$ in this test.
\endtab
\code


void fnpair_BitMatch1 (ffam_Fam *fam, fnpair_Res1 *res, fcho_Cho *cho,
                       long N, int r, int t,
                       int Nr, int j1, int j2, int jstep);
\endcode
\tab
 Similar to {\tt fnpair\_Bickel1} but with
 {\tt snpair\_ClosePairsBitMatch}.
\endtab
\code
\hide
#endif
\endhide
\endcode

