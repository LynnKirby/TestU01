\defmodule {utaus}

Implements simple and combined Tausworthe generators using
the definitions, the initialization methods and the
 algorithms given in \cite{rLEC94a,rLEC96a}.
The current implementation is restricted to components whose
characteristic polynomial is a {\em trinomial}.  That is,
for a  simple generator and for each component of a combined
generator, the basic recurrence has the form % (modulo 2)
\eq                                               \eqlabel{taus1}
  x_n = x_{n-r} \oplus x_{n-k} = x_{n-k+q} \oplus x_{n-k},
\endeq
with characteristic polynomial $p(x) = x^p + x^q + 1$,
where $q = k-r$, each $x_n$ is 0 or 1, and $\oplus$
means exclusive-or (i.e., addition modulo 2).
The output at step $n$ is
\eq                                               \eqlabel{taus2}
  u_n = \sum_{j=1}^w x_{ns+j-1} 2^{-j}
\endeq
with $w = 32$.
To obtain $w < 32$, it suffices to truncate the output.
The parameters must satisfy the following conditions:
$0 < 2q < k \le 32$ (except in the case of the {\tt LongTaus} generator
for which $k$ can take values as high as 64) and $0 < s \le r$.
In the functions defined below, the $k$ most significant
bits of the variable {\tt Y} contain
the initial values  $x_0,\dots,x_{k-1}$ (this is the seed).
They must not be all zero.
\index{Generator!Tausworthe} \index{Generator!LFSR}%

%%%%%%%%%%%%%%%%%%%%%%%%%%%%%%%%%%%%%%%%%%%%%%%%%%%%%%%%%%%%%%
\bigskip
\hrule
\code
\hide
/* utaus.h for ANSI C */

#ifndef UTAUS_H
#define UTAUS_H
\endhide
#include "TestU01/gdef.h"
#include "TestU01/unif01.h"


unif01_Gen * utaus_CreateTaus (unsigned int k, unsigned int q,
                               unsigned int s, unsigned int Y);
\endcode
  \tab  Implements a simple Tausworthe generator as described above.
   Restrictions: $0 < 2q < k \le 32$ and $0 < s \le k-q$.
  \endtab
\code


unif01_Gen * utaus_CreateTausJ (unsigned int k, unsigned int q,
                                unsigned int s, unsigned int j,
                                unsigned int Y);
\endcode
  \tab  Implements a Tausworthe generator as in {\tt utaus\_CreateTaus},
   except that it produces a $j$-decimated sequence.
   That is, at each call, it skips $j-1$ values in the sequence
   defined by (\ref{taus1}--\ref{taus2}) and outputs the next one.
   The same restrictions as in {\tt utaus\_CreateTaus} apply.
  \endtab
\code


unif01_Gen * utaus_CreateLongTaus (unsigned int k, unsigned int q,
                                   unsigned int s, uint64_t Y1);
\endcode
  \tab    Similar to {\tt utaus\_CreateTaus} but uses
   64 bits integers for the state of the generator. However, it returns
   only the 32 most significant bits of each  generated number, after
   having shifted them 32 bits to the right.
   Restrictions: $k \le 64$, $0 < 2q < k$ and $0 < s \le k-q$.
  \endtab
\code


unif01_Gen * utaus_CreateCombTaus2 (
   unsigned int k1, unsigned int k2, unsigned int q1, unsigned int q2,
   unsigned int s1, unsigned int s2, unsigned int Y1, unsigned int Y2);
\endcode
  \tab  Combines two Tausworthe generators defined as in
   {\tt utaus\_CreateTaus}. The combination is via a bitwise exclusive-or,
   as in \cite{rLEC96a,rTEZ89a,rTEZ91b}.
   The same restrictions as in  {\tt utaus\_CreateTaus} apply to each of the
   two components.
   Also assumes that $k_1\ge k_2$.
  \endtab
\code


unif01_Gen * utaus_CreateCombTaus3 (
    unsigned int k1, unsigned int k2, unsigned int k3,
    unsigned int q1, unsigned int q2, unsigned int q3,
    unsigned int s1, unsigned int s2, unsigned int s3,
    unsigned int Y1, unsigned int Y2, unsigned int Y3);
\endcode
  \tab  Similar to {\tt utaus\_CreateCombTaus2}, except that
   three Tausworthe generators are combined instead of two.
   Assumes that  $k_1 \ge k_2 \ge k_3$.
  \endtab
\code


unif01_Gen * utaus_CreateCombTaus3T (
    unsigned int k1, unsigned int k2, unsigned int k3,
    unsigned int q1, unsigned int q2, unsigned int q3,
    unsigned int s1, unsigned int s2, unsigned int s3,
    unsigned int Y1, unsigned int Y2, unsigned int Y3);
\endcode
  \tab  Similar to  {\tt utaus\_CreateCombTaus3}, except that the
   generator has ``triple'' precision.   Three successive output values
   $u_i$ of the combined Tausworthe generator are used to build each
   output value $U_i$ (uniform on [0, 1)) of this generator, as follows:
  $$
   U_{i} = \left(u_{3i} + \frac{u_{3i+1}}{2^{17}}  +
           \frac{u_{3i+2}}{2^{34}}\right) \mod 1.
  $$
  \endtab



\guisec{Clean-up functions}
\code

void utaus_DeleteGen (unif01_Gen *gen);
\endcode
 \tab \DelGen
 \endtab
\code
\hide
#endif
\endhide
\endcode


%%%%%%%%%%%%%%%%%%%%%%%%%%%%%%%%%%%%%%%%%%%%%%%%%%%%%%%%%%%%%%
\guisec{Related generators}


For specific Tausworthe generators, see also
{
\setlength{\partopsep}{0pt}
\setlength{\parskip}{0pt}
\setlength{\topsep}{0pt}
\setlength{\itemsep}{0pt}

\begin{itemize}
\item {\tt utezu\_CreateTezLec91}
\item {\tt utezu\_CreateTez95}
\end{itemize}

\bigskip
\begin{itemize}
\item {\tt ulec\_Createlfsr88}
\item {\tt ulec\_Createlfsr88T}
\item {\tt ulec\_Createlfsr113}
\item {\tt ulec\_Createlfsr258}
%\item {\tt ulec\_CreateCombTausLCG11}    %  Not a Tausworthe generator.
%\item {\tt ulec\_CreateCombTausLCG21}
\end{itemize}
}
